\documentclass{article}
\usepackage[utf8]{inputenc}
\usepackage{hevea}
\newcommand{\debcheck}{edos-debcheck}

\title{The EDOS-Debcheck Primer}
\author{Pietro Abate and Ralf Treinen and ...}

\begin{document}

\maketitle

The \debcheck{} tool determines, for a set of package control stanzas,
called \emph{the repository}, whether packages of the repository can
be installed relative to the repository. Typically, the repository is
a \texttt{Packages} file of a debian suite. The installability check
is by default performed for all package stanzas in the repository, but
may be also be restricted to subset.

\tableofcontents

\section{Packages and Repositories}
\label{sec:packages}
\subsection{Packages}
Debian control stanzas are defined in \texttt{deb-control (5)}. For
\debcheck{} only the following fields are relevant, all others are
ignored:
\begin{description}
\item[Package] giving the package name.
\item[Version] giving the version of the package.
\item[Architecture] specifying the architectures on which the package
  may be installed. Optional, defaults to \texttt{all}.
\item[Depends] is a list of items required for the installation of
  this package. Each item is a package name optionally with a version
  constraint, or a disjunction of these. Optional, defaults to the
  empty list.
\item[Pre-Depends] are by \debcheck{} treated like Depends.
\item[Conflicts] is a list of package names, possibly with a version
  constraint, that cannot be installed together with the package. Optional,
  defaults to the empty list.
\item[Breaks] are by \debcheck{} treated like Conflicts
\item[Provides] is a list of names symbolizing functionalities realized by the
  package. They have to be taken into account for dependencies and conflicts
  of other packages, see Section~\ref{sec:installability}.
\end{description}

In particular, \texttt{Recommends} and \texttt{Suggests} are ignored
by \debcheck.

** What is with Essential ??

\subsection{Repositories}
A \emph{repository} is a set of package stanzas. This set may be given
to \debcheck{} in form of a single file, or as several files (in the
latter case the repository is the set of all stanzas in all input
files). \debcheck{} makes two important assumptions on the
repositories; if any of these is not satisfied then it exits with an
error:

\begin{itemize}
\item
  We assume that there are no two package stanzas in the set that have
  both the same value of \texttt{Package} and the same value of
  \texttt{Version}. Having different versions for the same package
  name is OK, as it is of course OK to have two stanzas with different
  package names and the same version.

  As a consequence, the pair of package name and package version is a
  unique identifier of a package stanza inside a given repository. In
  the following, when we speak of \emph{a package}, we mean a precise
  package stanza that is identified by a name and a version, like the
  package of name \texttt{gcc} in version \texttt{4:4.3.2-2}; the
  stanza with name \texttt{gcc} and version \texttt{4:4.4.4-2} would
  constitute a different package.
\item
  We assume that the values of the \texttt{Architecture} fields are
  consistent, that is that there exists a single architecture that
  matches all these architecture values. Currently (policy version
  3.9.1), this means that all architecture values are either
  \texttt{all}, or the same specific architecture.

  When using the \texttt{-a \textit{<architecture>}} option, the
  repository is build from the input file(s) by filtering out all
  stanzas whose architecture value is not consistent with
  \textit{<architecture>} (see Section~\ref{sec:invocation}).
\end{itemize}

\section{Installability}
\label{sec:installability}

** explain peace, abundance, flatness, but in terms of debian packages and
relations with version constraints.

\subsection{What Installability does Not Mean}

** that the package installs cleanly (file overwrites, maintainer scripts,...)

** that any given tool like apt actually finds the solution 


\section{Invocation}
\label{sec:invocation}

\section{Output}
\label{sec:output}

\subsection{Understanding Explanations of Non-installability}

\subsection{Understanding Explanations of installability} 

\section{Credits}
\label{sec:credits}

** Jérôme as author of the solving machine, and of the first version
of the tool

** EDOS, Mancoosi projects 

\section{Further Reading}
** Relation to CUDF
 
\end{document}

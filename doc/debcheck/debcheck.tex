\documentclass{article}
\usepackage[utf8]{inputenc}
\usepackage{hevea}
\newcommand{\debcheck}{edos-debcheck}
\newenvironment{example}{\par\textbf{Example:}}{\par}

\title{The EDOS-Debcheck Primer}
\author{Pietro Abate and Ralf Treinen and ...}

\begin{document}

\maketitle

The \debcheck{} tool determines, for a set of package control stanzas,
called \emph{the repository}, whether packages of the repository can
be installed relative to the repository. Typically, the repository is
a \texttt{Packages} file of a debian suite. The installability check
is by default performed for all package stanzas in the repository, but
may be also be restricted to a subset of these.

\tableofcontents

\section{Input Data: Packages and Repositories}
\label{sec:data}
\subsection{Packages}
Debian control stanzas are defined in \texttt{deb-control (5)}. For
\debcheck{} only the following fields are relevant, all others are
ignored:
\begin{description}
\item[Package] giving the package name.
\item[Version] giving the version of the package.
\item[Architecture] specifying the architectures on which the package
  may be installed. Optional, defaults to \texttt{all}.
\item[Depends] is a list of items required for the installation of
  this package. Each item is a package name optionally with a version
  constraint, or a disjunction of these. Optional, defaults to the
  empty list.
\item[Pre-Depends] are by \debcheck{} treated like Depends.
\item[Conflicts] is a list of package names, possibly with a version
  constraint, that cannot be installed together with the package. Optional,
  defaults to the empty list.
\item[Breaks] are by \debcheck{} treated like Conflicts
\item[Provides] is a list of names symbolizing functionalities realized by the
  package. They have to be taken into account for dependencies and conflicts
  of other packages, see Section~\ref{sec:installability}.
\end{description}

In particular, \texttt{Recommends} and \texttt{Suggests} are ignored
by \debcheck. Also, \debcheck{} does not check for the existance of
fields that are required by Debian policy but that are not relevant
for the task of \debcheck, like \texttt{Maintainer} or
\texttt{Description}. *** Is this true?

** What is with the Essential field ??

\subsection{Repositories}
A \emph{repository} is a set of package stanzas. This set may be given
to \debcheck{} in form of a single file or as several files, in the
latter case the repository is constituted by all stanzas in all input
files (see Section~\ref{sec:invocation}). \debcheck{} makes two
important assumptions on the repositories; if any of these is not
satisfied then it exits with an error:

\begin{itemize}
\item
  We assume that there are no two package stanzas in the repository
  that have both the same value of \texttt{Package} and the same value
  of \texttt{Version}. Having different versions for the same package
  name is OK, as it is of course OK to have two stanzas with different
  package names and the same version.

  As a consequence, the pair of package name and package version is a
  unique identifier of a package stanza inside a given repository. In
  the following, when we speak of \emph{a package}, we mean a precise
  package stanza that is identified by a name \emph{and} a version,
  like the package of name \texttt{gcc} in version \texttt{4:4.3.2-2}.
  The stanza with name \texttt{gcc} and version \texttt{4:4.4.4-2}
  would constitute a different package.
\item
  We assume that the values of the \texttt{Architecture} fields are
  consistent, that is that there exists a single architecture that
  matches all these architecture values. Currently (policy version
  3.9.1), this means that all architecture values are either
  \texttt{all}, or the same specific architecture.

  An exception is the \texttt{-a \textit{<architecture>}} option which
  allows to restrict the repository to those package stanzas who's
  architecture field is consistent with a given concrete architecture
  (see Section~\ref{sec:invocation}).
\end{itemize}

\section{Installability}
\label{sec:installability}

Given a repository (see Section~\ref{sec:data}) $R$, we define an
$R$-installation set a subset $I$ of $R$ that has the following three
properties:
\begin{description}
  \item[flatness:] $I$ does not contain two different packages with the same
    name (which then would have different versions).
  \item[abundance:] For each package in $I$, every of its dependencies
    is satisfied by some package in $I$, either directly or through a
    virtual package in case the dependency does not carry a version
    constraint.
  \item[peace:] For each package in $I$ and for each item in its list
    of conflicts, no package in $I$ satisfies the description of that
    item.  As an exception, it is allowed that a package in $I$ both
    provides a virtual package and at the same time conflicts with it.
\end{description}
Hence, the notion of an installation captures the idea that a certain
set of packages may be installed together on a machine, following the
semantics of binary package relations according to the Debian Policy.

** What is with Essential ??

\begin{example}
  Let $R$ be the following repository:
\begin{verbatim}
    Package: a
    Version: 1
    Depends: b (>= 2) | v

    Package: a 
    Version: 2
    Depends: c (> 1)

    Package: b
    Version: 1
    Conflicts: d

    Package: c
    Version: 3
    Depends: d
    Conflicts: v

    Package: d
    Version: 5
    Provides: v
    Conflicts: v
\end{verbatim}

The following subsets of $R$ are not $R$-installation sets:
\begin{itemize}
\item The complete set $R$ since it is not flat (it contains two
  different packages with name $a$)
\item The set $\{(a,1), (c,3)\}$ since it not abundant (the dependency
  of $(a,1)$ is not satisfied, nor is the dependency of $(c,3)$).
\item The set $\{(a,2), (c,3), (d,5)\}$ since it is not in peace
  (there is conflict between $(c,3)$ and $(d,5)$ via the virtual package $v$)
\end{itemize}
Examples of $R$-installation sets are
\begin{itemize}
\item The set $\{(d,5)\}$ (self conflicts via virtual packages are ignored)
\item The set $\{(a,1), (b,1)\}$
\item The set $\{(a,1), (d,5)\}$
\end{itemize}
\end{example}

A package $(p,n)$ is said to be \emph{installable} in a repository $R$
if there exists an $R$-installation set $I$ that contains $(p,n)$.

\begin{example}
  In the above example, $(a,1)$ is $R$-installable since it is contained
  in the $R$-installation set $\{(a,1), (d,5) \}$.

  However, $(a,2)$ is not $R$-installable: Any $R$-installation set
  containing $(a,2)$ must also contain $(c,3)$ (since it is the only
  package in $R$ that can satisfy the dependency of $(a,2)$ on $c
  (>1)$, and in the same way it must also contain $(d,5)$. However, this
  destroys the peace as $(c,3)$ and $(d,5)$ are in conflict. Hence, no such
  $R$-installation set can exist.
\end{example}

\subsection{What Installability does Not Mean}

\begin{itemize}
\item Installability in the sense of \debcheck{} only concerns the
  relations between different binary packages expressed in their
  respective control files. It does not mean that a package indeed
  installs cleanly in a particular environment since an installation
  attempt may still fail for different reasons, like failure of a
  maintainer script or attempting to hijack a file owned by another
  already installed package.
\item Installability means theoretical existence of a solution. It
  does not mean that a package manager (like \texttt{aptitude},
  \texttt{apt-get}) actually finds a way to install that package.
  This failure to find a solution may be due to an inherent
  incompleteness of the dependency resolution algorithm employed by
  the package manager, or may be due to user-defined preferences that
  exclude certain solutions.
\end{itemize}

One also should keep in mind that, even when two packages are
$R$-installable, this does not necessarily mean that both packages can
be installed \emph{together}. A set $P$ of packages is called
$R$-\emph{co-installable} when there exists a single $R$-installation
set extending $P$.

\begin{example}
  Again in the above example, both $(b,1)$ and $(d,5)$ are
  $R$-installable; however they are not $R$-co-installable.
\end{example}

See Section~\ref{sec:tricks} on how co-installability can be encoded.


\section{Invocation}
\label{sec:invocation}

\section{Output}
\label{sec:output}

\subsection{Understanding Explanations of Non-installability}

\subsection{Understanding Explanations of installability} 

** We should fix teh error codes (also for inconsistent input data)

\section{Tips and Tricks}
\label{sec:tricks}
** Explain how to encode a complicated query (installing several
packages togther) by creating a pseudo-package

\section{Credits}
\label{sec:credits}

** Jérôme as author of the solving machine, and of the first version
of the tool

** EDOS, Mancoosi projects 

\section{Further Reading}
** Relation to CUDF
 
\end{document}

%%%%%%%%%%%%%%%%%%%%%%%%%%%%%%%%%%%%%%%%%%%%%%%%%%%%%%%%%%%%%%%%%%%%%%%%%%
%  Copyright (C) 2010-2011  Pietro Abate <pietro.abate@pps.jussieu.fr>   %
%                           Ralf Treinen <ralf.treinen@pps.jussieu.fr>   %
%                           Unversité Paris-Diderot                      %
%                                                                        %
%  This documentation is free software: you can redistribute it and/or   %
%  modify it under the terms of the GNU General Public License as        %
%  published by the Free Software Foundation, either version 3 of the    %
%  License, or (at your option) any later version.                       %
%%%%%%%%%%%%%%%%%%%%%%%%%%%%%%%%%%%%%%%%%%%%%%%%%%%%%%%%%%%%%%%%%%%%%%%%%%

\documentclass{article}
\usepackage[utf8]{inputenc}
\usepackage{hevea}
\newcommand{\debcheck}{dose-debcheck}
\newcommand{\distcheck}{dose-distcheck}
\newenvironment{example}%
  {\par\begin{divstyle}{example}\par\textbf{Example:}}%
  {\end{divstyle}\par}
\newstyle{.example}{border:solid black;background:\#eeddbb;padding:lex}

\title{The Dose-Debcheck Primer}
\author{Pietro Abate and Ralf Treinen}

\begin{document}

\maketitle

The \debcheck{} tool determines, for a set of package control stanzas,
called \emph{the repository}, whether packages of the repository can
be installed relative to the repository. Typically, the repository is
a \texttt{Packages} file of a debian suite. The installability check
is by default performed for all package stanzas in the repository, but
may be also be restricted to a subset of these.

\tableofcontents

\section{Input Data: Packages and Repositories}
\label{sec:data}
\subsection{Packages}
Debian control stanzas are defined in \texttt{deb-control (5)}. For
\debcheck{} only the following fields are relevant, all others are
ignored:
\begin{description}
\item[Package] giving the package name.
\item[Version] giving the version of the package.
\item[Architecture] specifying the architectures on which the package
  may be installed. Optional, defaults to \texttt{all}.
\item[Depends] is a list of items required for the installation of
  this package. Each item is a package name optionally with a version
  constraint, or a disjunction of these. Optional, defaults to the
  empty list.
\item[Pre-Depends] are by \debcheck{} treated like Depends.
\item[Conflicts] is a list of package names, possibly with a version
  constraint, that cannot be installed together with the package. Optional,
  defaults to the empty list.
\item[Breaks] are by \debcheck{} treated like Conflicts
\item[Provides] is a list of names symbolizing functionalities realized by the
  package. They have to be taken into account for dependencies and conflicts
  of other packages, see Section~\ref{sec:installability}.
\end{description}

In particular, \texttt{Recommends} and \texttt{Suggests} are ignored
by \debcheck. Also, \debcheck{} does not check for the existance of
fields that are required by Debian policy but that are not relevant
for the task of \debcheck, like \texttt{Maintainer} or
\texttt{Description}. *** Is this true?

** What is with the Essential field ??

\subsection{Repositories}
A \emph{repository} is a set of package stanzas. This set may be given
to \debcheck{} in form of a single file or as several files, in the
latter case the repository is constituted by all stanzas in all input
files (see Section~\ref{sec:invocation}). \debcheck{} makes two
important assumptions on the repositories; if any of these is not
satisfied then it exits with an error:

\begin{itemize}
\item
  We assume that there are no two package stanzas in the repository
  that have both the same value of \texttt{Package} and the same value
  of \texttt{Version}. Having different versions for the same package
  name is OK, as it is of course OK to have two stanzas with different
  package names and the same version.

  As a consequence, the pair of package name and package version is a
  unique identifier of a package stanza inside a given repository. In
  the following, when we speak of \emph{a package}, we mean a precise
  package stanza that is identified by a name \emph{and} a version,
  like the package of name \texttt{gcc} in version \texttt{4:4.3.2-2}.
  The stanza with name \texttt{gcc} and version \texttt{4:4.4.4-2}
  would constitute a different package.
\item
  We assume that the values of the \texttt{Architecture} fields are
  consistent, that is that there exists a single architecture that
  matches all these architecture values. Currently (policy version
  3.9.1), this means that all architecture values are either
  \texttt{all}, or the same specific architecture.

  An exception is the \texttt{-a \textit{<architecture>}} option which
  allows to restrict the repository to those package stanzas who's
  architecture field is consistent with a given concrete architecture
  (see Section~\ref{sec:invocation}).
\end{itemize}

\section{Installability}
\label{sec:installability}

Given a repository (see Section~\ref{sec:data}) $R$, we define an
$R$-installation set a subset $I$ of $R$ that has the following three
properties:
\begin{description}
  \item[flatness:] $I$ does not contain two different packages with the same
    name (which then would have different versions).
  \item[abundance:] For each package in $I$, every of its dependencies
    is satisfied by some package in $I$, either directly or through a
    virtual package in case the dependency does not carry a version
    constraint.
  \item[peace:] For each package in $I$ and for each item in its list
    of conflicts, no package in $I$ satisfies the description of that
    item.  As an exception, it is allowed that a package in $I$ both
    provides a virtual package and at the same time conflicts with it.
\end{description}
Hence, the notion of an installation captures the idea that a certain
set of packages may be installed together on a machine, following the
semantics of binary package relations according to the Debian Policy.

** What is with Essential ??

\begin{example}
  Let $R$ be the following repository:
\begin{verbatim}
    Package: a
    Version: 1
    Depends: b (>= 2) | v

    Package: a 
    Version: 2
    Depends: c (> 1)

    Package: b
    Version: 1
    Conflicts: d

    Package: c
    Version: 3
    Depends: d
    Conflicts: v

    Package: d
    Version: 5
    Provides: v
    Conflicts: v
\end{verbatim}

The following subsets of $R$ are not $R$-installation sets:
\begin{itemize}
\item The complete set $R$ since it is not flat (it contains two
  different packages with name $a$)
\item The set $\{(a,1), (c,3)\}$ since it not abundant (the dependency
  of $(a,1)$ is not satisfied, nor is the dependency of $(c,3)$).
\item The set $\{(a,2), (c,3), (d,5)\}$ since it is not in peace
  (there is conflict between $(c,3)$ and $(d,5)$ via the virtual package $v$)
\end{itemize}
Examples of $R$-installation sets are
\begin{itemize}
\item The set $\{(d,5)\}$ (self conflicts via virtual packages are ignored)
\item The set $\{(a,1), (b,1)\}$
\item The set $\{(a,1), (d,5)\}$
\end{itemize}
\end{example}

A package $(p,n)$ is said to be \emph{installable} in a repository $R$
if there exists an $R$-installation set $I$ that contains $(p,n)$.

\begin{example}
  In the above example, $(a,1)$ is $R$-installable since it is contained
  in the $R$-installation set $\{(a,1), (d,5) \}$.

  However, $(a,2)$ is not $R$-installable: Any $R$-installation set
  containing $(a,2)$ must also contain $(c,3)$ (since it is the only
  package in $R$ that can satisfy the dependency of $(a,2)$ on $c
  (>1)$, and in the same way it must also contain $(d,5)$. However, this
  destroys the peace as $(c,3)$ and $(d,5)$ are in conflict. Hence, no such
  $R$-installation set can exist.
\end{example}

\subsection{What Installability does Not Mean}

\begin{itemize}
\item Installability in the sense of \debcheck{} only concerns the
  relations between different binary packages expressed in their
  respective control files. It does not mean that a package indeed
  installs cleanly in a particular environment since an installation
  attempt may still fail for different reasons, like failure of a
  maintainer script or attempting to hijack a file owned by another
  already installed package.
\item Installability means theoretical existence of a solution. It
  does not mean that a package manager (like \texttt{aptitude},
  \texttt{apt-get}) actually finds a way to install that package.
  This failure to find a solution may be due to an inherent
  incompleteness of the dependency resolution algorithm employed by
  the package manager, or may be due to user-defined preferences that
  exclude certain solutions.
\end{itemize}

One also should keep in mind that, even when two packages are
$R$-installable, this does not necessarily mean that both packages can
be installed \emph{together}. A set $P$ of packages is called
$R$-\emph{co-installable} when there exists a single $R$-installation
set extending $P$.

\begin{example}
  Again in the above example, both $(b,1)$ and $(d,5)$ are
  $R$-installable; however they are not $R$-co-installable.
\end{example}

See Section~\ref{sec:tricks} on how co-installability can be encoded.

%%%%%%%%%%%%%%%%%%%%%%%%%%%%%%%%%%%%%%%%%%%%%%%%%%%%%%%%%%%%%%%%%%%%%%%%%%
%  Copyright (C) 2010-2012  Pietro Abate <pietro.abate@pps.jussieu.fr>   %
%                           Ralf Treinen <ralf.treinen@pps.jussieu.fr>   %
%                           Unversité Paris-Diderot                      %
%                                                                        %
%  This documentation is free software: you can redistribute it and/or   %
%  modify it under the terms of the GNU General Public License as        %
%  published by the Free Software Foundation, either version 3 of the    %
%  License, or (at your option) any later version.                       %
%%%%%%%%%%%%%%%%%%%%%%%%%%%%%%%%%%%%%%%%%%%%%%%%%%%%%%%%%%%%%%%%%%%%%%%%%%

\section{Invocation}
\label{sec:invocation}

\subsection{Basic usage}

\debcheck{} accepts several different options, and also arguments.

\begin{alltt}
  \debcheck{} [option] ... [file] ...
\end{alltt}

The package repository is partionend into a \emph{background} and a
\emph{foreground}. The foreground contains the packages we are actually
interested in, the background contains packages that are just available
for satisfying dependencies, but for which we do not care about installability.

All arguments are interpreted as filenames of Packages input files,
the contents of which go into the foreground. If no argument is given
then metadata of foreground packages is read from standard input.  In
addition, one may specify listings of foreground packages with the
option \verb|--fg=<filename>|, and listings of background packages
with the option \verb|--bg=<filename>|. Input from files (but not from
standard input) may be compressed with gzip or bzip2, provided
\debcheck{} was compiled with support for these compression libraries.

The option \texttt{-f} and \texttt{-s} ask for a listing of uninstallable,
resp.\ installable packages. The option \texttt{-e} asks for an explanation
of each reported case. The exact effect of these options will be explained
in Section~\ref{sec:output}.

\begin{example}
We may check whether packages in \textit{non-free} are installable,
where dependencies may be satisfied from \textit{main} or \textit{contrib}:
\begin{verbatim}
dose-distcheck  -f -e \
    --bg=/var/lib/apt/lists/ftp.fr.debian.org_debian_dists_sid_main_binary-amd64_Packages\
    --bg=/var/lib/apt/lists/ftp.fr.debian.org_debian_dists_sid_contrib_binary-amd64_Packages\
    /var/lib/apt/lists/ftp.fr.debian.org_debian_dists_sid_non-free_binary-amd64_Packages
\end{verbatim}
\end{example} 

\subsection{Checking only selected packages}
\label{sec:invocation-background}
The initial distinction between foreground and background packages is
modified when using the \verb|--checkonly| option. This option takes
as value a comma-separated list of package names, possibly qualified
with a version constraint. The effect is that only packages that match
one of these package names are kept in the foreground, all others are
pushed into the background.

\begin{example}
\begin{alltt}
\debcheck{} --checkonly "libc6, 2ping (= 1.2.3-1)" Packages
\end{alltt}
\end{example}

\subsection{Checking for co-installability}
\label{sec:invocation-coinst}
Co-installability of packages can be easily checked with the
\verb|--coinst| option. This option takes as argument a
comma-separated list of packages, each of them possibly with a version
constraint. In that case, \debcheck{} will check whether the packages
specified are co-installable, that is whether it is possible to
install these packages at the same time (see
Section~\ref{sec:coinstallability}).

Note that it is possible that the name of a package, even when
qualified with a version constraint, might be matched by several
packages with different versions. In that case, co-installability will
be checked for \emph{each} combination of real packages that match the
packages specified in the argument of the \verb|--coinst| option.
\begin{example}
  Consider the following repository (architectures are omitted for
  clarity):
\begin{verbatim}
Package: a
Version: 1

Package: a 
Version: 2

Package: a
Version: 3

Package: b
Version: 10

Package: b
Version: 11

...
\end{verbatim}
Executing the command \verb|debcheck --coinst a (>1), b| on this
repository will check co-installability of 4 pairs of packages: there
are two packages that match \verb|a (>1)|, namely package \texttt{a} in
versions 2 and 3, and there are two packages that match \texttt{b}. Hence,
the following four pairs of packages will be checked for co-installability:
\begin{enumerate}
\item (a,2), (b,10)
\item (a,2), (b,11)
\item (a,3), (b,10)
\item (a,3), (b,11)
\end{enumerate}
\end{example}

Mathematically speaking, the set of checked tuples is the Cartesian product
of the denotations of the single package specifications.

\subsection{Changing the Notion of Installability}

Some options affect the notion of installability:
\begin{itemize}
\item \texttt{--deb-ignore-essential} drops the Foundation requirement
  of installation sets (Section~\ref{sec:installability}). In other
  words, it is no longer required that any installation set contains all
  essential packages.
\end{itemize}

Other options concern Multiarch:
\begin{itemize}
\item \texttt{--deb-native-arch=}\textit{a} sets the native
  architecture to the value $a$. Note that the native architecture is
  not necessarily the architecture on which the tool is executed, it
  is just the primary architecture for which we are checking
  installability of packages. In particular, packages with the
  architecture field set to \texttt{all} are interpreted as packages of the
  native architecture \cite{ubuntu:multiarch}.
\item \texttt{--deb-foreign-archs=}$a_1,\ldots,a_n$ sets the foreign
  architectures to the list $a_1,\ldots,a_n$. Packages may only be installed
  when their architecture is the native architecture (including \texttt{all}),
  or one of the foreign architectures.
\end{itemize}


\subsection{Filtering Packages and Multiarch}
Filtering out packages is a different operation than pushing packages
into the background (Section~\ref{sec:invocation-background}): Background
packages are still available to satisfy dependencies, while filtering out a 
package makes it completely invisible.

\begin{itemize}
\item The effect of \texttt{--latest} is to keep only the latest version of any
package.
\end{itemize}


\subsection{Other Options}
Other options controlling the output are explained in detail in
Section~\ref{sec:output}. A complete listing of all options can be found in
the \debcheck(1) manpage.



%%%%%%%%%%%%%%%%%%%%%%%%%%%%%%%%%%%%%%%%%%%%%%%%%%%%%%%%%%%%%%%%%%%%%%%%%%
%  Copyright (C) 2010, 2011 Pietro Abate <pietro.abate@pps.jussieu.fr>   %
%                           Ralf Treinen <ralf.treinen@pps.jussieu.fr>   %
%                           Unversité Paris-Diderot                     %
%                                                                        %
%  This documentation is free software: you can redistribute it and/or   %
%  modify it under the terms of the GNU General Public License as        %
%  published by the Free Software Foundation, either version 3 of the    %
%  License, or (at your option) any later version.                       %
%%%%%%%%%%%%%%%%%%%%%%%%%%%%%%%%%%%%%%%%%%%%%%%%%%%%%%%%%%%%%%%%%%%%%%%%%%

\section{Output}
\label{sec:output}

\subsection{Understanding Explanations of Non-installability}

A package can be broken because of a missing package or because of a
conflict. For a missing package we'll have a stanza like this :

\begin{verbatim}
package: libgnuradio-dev
  version: 3.2.2.dfsg-1
  architecture: all
  source: gnuradio (= 3.2.2.dfsg-1)
  status: broken
  reasons:
   -
    missing:
     pkg:
      package: libgruel0
      version: 3.2.2.dfsg-1+b1
      architecture: amd64
      missingdep: libboost-thread1.40.0 (>= 1.40.0-1)
     paths:
      -
       depchain:
        -
         package: libgnuradio-dev
         version: 3.2.2.dfsg-1
         architecture: all
         depends: libgnuradio (= 3.2.2.dfsg-1)
        -
         package: libgnuradio
         version: 3.2.2.dfsg-1
         architecture: all
         depends: libgnuradio-core0
        -
         package: libgnuradio-core0
         version: 3.2.2.dfsg-1+b1
         architecture: amd64
         depends: libgruel0 (= 3.2.2.dfsg-1+b1)
\end{verbatim}

The first part gives details about the package libgnuradio-dev, specifying its
status, source and architecture. The second part is the reason of the problem.
In this case it is a missing package that is essential to install
libgnuradio-dev. missindep is the dependency that cannot be satisfied is the
package libgruel0 , in this case: {\tt libboost-thread1.40.0 (>= 1.40.0-1)}.

The paths component gives all possible depchains from the root package
libgnuradio-dev to libgruel0 . Notice that we do not include the last node in
the dependency chain to avoid a useless repetition. Of course there might be
more then on path to reach libgruel0. Distcheck will unroll all of them.
Because of the structure of debian dependencies usually there are not so many
paths.

\begin{verbatim}
package: a
version: 1
depends: b, d

package: b
version: 1
depends: e

package: d
version: 1
depends: f

package: f
version: 1
conflicts: e

package: e
version: 1
conflicts: f
\end{verbatim}

The other possible cause of a problem is a conflict. Consider the following :

\begin{verbatim}
package: a
  version: 1
  status: broken
  reasons:
   -
    conflict:
     pkg1:
      package: e
      version: 1
     pkg2:
      package: f
      version: 1
     depchain1:
      -
       depchain:
        -
         package: a
         version: 1
         depends: b
        -
         package: b
         version: 1
         depends: e
     depchain2:
      -
       depchain:
        -
         package: a
         version: 1
         depends: d
        -
         package: d
         version: 1
         depends: f
\end{verbatim}

This is the general case of a deep conflict. I use an artificial example here
instead of a concrete one since this case is not very common and I was not able
to find one.

The first part of the distcheck report is as before with details about the
broken package. Since this is a conflict, and all conflicts are binary, we give
the two packages involved in the conflict first. Packages f and e are in
conflict, but they are not direct dependency of package a . For this reason, we
output the two paths that from a lead to f or e. All dependency chains for each
conflict are together. Again, since there might be more than one way from a to
reach the conflicting packages, we can have more then one depchain. 

\subsection{Understanding Explanations of installability}

%** We should fix the error codes (also for inconsistent input data)


%%%%%%%%%%%%%%%%%%%%%%%%%%%%%%%%%%%%%%%%%%%%%%%%%%%%%%%%%%%%%%%%%%%%%%%%%%
%  Copyright (C) 2010-2012  Pietro Abate <pietro.abate@pps.jussieu.fr>   %
%                           Ralf Treinen <ralf.treinen@pps.jussieu.fr>   %
%                           Unversité Paris-Diderot                      %
%                                                                        %
%  This documentation is free software: you can redistribute it and/or   %
%  modify it under the terms of the GNU General Public License as        %
%  published by the Free Software Foundation, either version 3 of the    %
%  License, or (at your option) any later version.                       %
%%%%%%%%%%%%%%%%%%%%%%%%%%%%%%%%%%%%%%%%%%%%%%%%%%%%%%%%%%%%%%%%%%%%%%%%%%

\section{Tips and Tricks}
\label{sec:tricks}
\subsection{Encoding checks involving several packages}
\debcheck{} only tests whether any package in the foreground set is
installable. However, sometimes one is interested in knowing whether
several packages are co-installable, that is whether there exists an
installation set that contains all these packages. One might also be
interested in an installation that does \emph{not} contain a certain
package.

This can be encoded by creating a pseudo-package that
represents the query. 

\begin{example}
  We wish to know whether it is possible to install at the same time
  \texttt{a} and \texttt{b}, the latter in some version $\geq 42$, but
  without installing c. We create a pseudo package like this:
\begin{verbatim}
Package: query
Version: 1
Architecture: all
Depends: a, b(>= 42)
Conflicts: c
\end{verbatim}
Then we check for installability of that package with respect to the
repository:
\begin{verbatim}
echo "Package: query\nVersion: 1\nArchitecture: all\nDepends: a, b(>=42)\nConflicts: c" | dose-debcheck --bg=repository
\end{verbatim}
(Beware: This might not do exactly what you want, see below!)
\end{example}

The problem with this encoding is as follows: if we ask \debcheck{}
for installability of some package depending on \texttt{a} then this
dependency can a priori be satisfied by any of the available versions
of package \texttt{a}, or even by some other package that provides
\texttt{a} as a virtual package. Virtual packages can be excluded by
exploiting the fact that, in Debian, virtual packages are not
versioned. As a consequence, any package relation (like Depends)
containing a version constraint can only be matched by a real package,
and not by a virtual package. This means that the dependency on
\texttt{b (>= 42)} in the above example already can only be matched by
a real package. If we also want to restrict dependency on \texttt{a}
to real packages only without knowing its possible versions, then we
may write \texttt{Depends: a (>=0) | a(<0)}.

\begin{example}
  If we wish to know whether it is possible to install at the same
  time some version of package \texttt{a} and some version of package
  \texttt{b}, under the condition that these are real packages and not
  virtual packages, then we may construct the following pseudo-package
  and check its installability:
\begin{verbatim}
Package: query
Version: 1
Architecture: all
Depends: a(>=0) | a(<0), b(>=0) | b(<0)
\end{verbatim}
\end{example}

Note that it is in theory possible, though admittedly quite unlikely,
that a package has a version number smaller than $0$ (example:
$0\sim$).

However, if we have several versions of package \texttt{a} and several
versions of package \texttt{b} then the above pseudo-package is
installable if it is possible to install at the same time \emph{some
  version} of \texttt{a} and \emph{some version} of \texttt{b}. If we
want instead to check co-installability of any combination of versions
of package \texttt{a} with versions of package \texttt{b} then the
\texttt{--coinst} option is better suited for the task.

\subsection{Parsing \debcheck's output in Python}
\label{sec:tricks-python}
Debcheck's output can be easily parsed from a Python program by using
the YAML parser (needs the Debian package \texttt{python-yaml}).

\begin{example}
  If you have run debcheck with the option \texttt{-f} (and possibly
  with the \texttt{-s} option in addition) you may obtain a report
  containing one non-installable package (name and version) per line
  like this:
  
\begin{verbatim}
import yaml

doc = yaml.load(file('output-of-distcheck', 'r'))
if doc['report'] is not None:
  for p in doc['report']:
    if p['status'] == 'broken':
      print '%s %s is broken' (p['package'], p['version'])
\end{verbatim}
\end{example}

A complete example of a python script that constructs a set of
pseudo-packages, runs \debcheck{} on it, and then processes the output
is given in the directory
\texttt{doc/examples/potential-file-overwrites}.

\subsection{Usage as a test in a shell script}
Exit codes allow for a convenient integration of installation checks
as tests in shell scripts.

\begin{example}
Suppose that you want to check installability of all \verb|.deb| files
in the current directory with respect to the repository
\verb|unstable.packages| before uploading your package described in
\verb|mypackage.changes|:

\begin{verbatim}
find . -name "*.deb" -exec dpkg-deb --info '{}' control \; -exec echo ""\; | \
  dose-debcheck --bg unstable.packages && dput mypackage.changes
\end{verbatim}
\end{example}

%%%%%%%%%%%%%%%%%%%%%%%%%%%%%%%%%%%%%%%%%%%%%%%%%%%%%%%%%%%%%%%%%%%%%%%%%%
%  Copyright (C) 2010, 2010 Pietro Abate <pietro.abate@pps.jussieu.fr>   %
%                           Ralf Treinen <ralf.treinen@pps.jussieu.fr>   %
%                           Unversité Paris-Diderot                     %
%                                                                        %
%  This documentation is free software: you can redistribute it and/or   %
%  modify it under the terms of the GNU General Public License as        %
%  published by the Free Software Foundation, either version 3 of the    %
%  License, or (at your option) any later version.                       %
%%%%%%%%%%%%%%%%%%%%%%%%%%%%%%%%%%%%%%%%%%%%%%%%%%%%%%%%%%%%%%%%%%%%%%%%%%

\section{Credits}
\label{sec:credits}

Jérôme Vouillon is the author of the solving engine. He also wrote the
first version of the program (called \textsc{debcheck} and
\textsc{rpmcheck} at that time), which was released in November 2005.

The initial development of this tool was supported by the research
projects \emph{Environment for the development and Distribution of
  Open Source software (EDOS)}, funded by the European Commission
under the IST activities of the 6th Framework Programme. Further
development and maintenance of the software, together with new
applications building on top of it, are funded by the research project
\emph{Managing the Complexity of the Open Source Infrastructure
  (Mancoosi)}, funded by the European Commission under the IST
activities of the 6th Framework Programme, grant agreement 214898.

The tool, the underlying theory and its application, was described in
\cite{edos2006ase}.


%%%%%%%%%%%%%%%%%%%%%%%%%%%%%%%%%%%%%%%%%%%%%%%%%%%%%%%%%%%%%%%%%%%%%%%%%%
%  Copyright (C) 2010-2012  Pietro Abate <pietro.abate@pps.jussieu.fr>   %
%                           Ralf Treinen <ralf.treinen@pps.jussieu.fr>   %
%                           Unversité Paris-Diderot                      %
%                                                                        %
%  This documentation is free software: you can redistribute it and/or   %
%  modify it under the terms of the GNU General Public License as        %
%  published by the Free Software Foundation, either version 3 of the    %
%  License, or (at your option) any later version.                       %
%%%%%%%%%%%%%%%%%%%%%%%%%%%%%%%%%%%%%%%%%%%%%%%%%%%%%%%%%%%%%%%%%%%%%%%%%%

\section{Further Reading}
** Relation to CUDF 

\section{Copyright}
\label{sec:copyright}
This documentation is Copyright \copyright {} 2010, 2011 Pietro Abate
\verb|<pietro.abate@pps.jussieu.fr>| and Ralf Treinen
\verb|<ralf.treinen@pps.jussieu.fr>|.


\bibliographystyle{alpha}
\bibliography{mancoosi}

\end{document}

%%%%%%%%%%%%%%%%%%%%%%%%%%%%%%%%%%%%%%%%%%%%%%%%%%%%%%%%%%%%%%%%%%%%%%%%%%
%  Copyright (C) 2010, 2011 Pietro Abate <pietro.abate@pps.jussieu.fr>   %
%                           Ralf Treinen <ralf.treinen@pps.jussieu.fr>   %
%                           Unversité Paris-Diderot                     %
%                                                                        %
%  This documentation is free software: you can redistribute it and/or   %
%  modify it under the terms of the GNU General Public License as        %
%  published by the Free Software Foundation, either version 3 of the    %
%  License, or (at your option) any later version.                       %
%%%%%%%%%%%%%%%%%%%%%%%%%%%%%%%%%%%%%%%%%%%%%%%%%%%%%%%%%%%%%%%%%%%%%%%%%%

\section{Tips and Tricks}
\label{sec:tricks}
\subsection{Encoding checks involving several package}
\debcheck{} only tests whether any package in the foreground set is
installable. However, sometimes one is interested in knowing whether
several packages are co-installable, that is whether there exists an
an installation set that contains all these packages. One might also
be interested in an installation that does \emph{not} contain a certain
package.

This can be encoded by creating a pseudo-package that
represents the query. 

\begin{example}
  We wish to know whether it is possible to install at the same time
  \texttt{a} and \texttt{b}, the latter in some version $\geq 42$, but
  without installing c. We create a pseudo package like this:
\begin{verbatim}
Package: query
Depends: a, b(>= 42)
Conflicts: c
\end{verbatim}
Then we check for installability of that package with respect to the
repository:
\begin{verbatim}
echo "Package: query\nDepends: a, b(>=42)\nConflicts: c" | debcheck -b repository --
\end{verbatim}
\end{example}



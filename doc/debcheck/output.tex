%%%%%%%%%%%%%%%%%%%%%%%%%%%%%%%%%%%%%%%%%%%%%%%%%%%%%%%%%%%%%%%%%%%%%%%%%%
%  Copyright (C) 2010, 2011 Pietro Abate <pietro.abate@pps.jussieu.fr>   %
%                           Ralf Treinen <ralf.treinen@pps.jussieu.fr>   %
%                           Unversité Paris-Diderot                     %
%                                                                        %
%  This documentation is free software: you can redistribute it and/or   %
%  modify it under the terms of the GNU General Public License as        %
%  published by the Free Software Foundation, either version 3 of the    %
%  License, or (at your option) any later version.                       %
%%%%%%%%%%%%%%%%%%%%%%%%%%%%%%%%%%%%%%%%%%%%%%%%%%%%%%%%%%%%%%%%%%%%%%%%%%

\section{Output}
\label{sec:output}

\subsection{Understanding Explanations of Non-installability}

A package can be broken because of a missing package or because of a
conflict. For a missing package we'll have a stanza like this :

\begin{verbatim}
package: libgnuradio-dev
  version: 3.2.2.dfsg-1
  architecture: all
  source: gnuradio (= 3.2.2.dfsg-1)
  status: broken
  reasons:
   -
    missing:
     pkg:
      package: libgruel0
      version: 3.2.2.dfsg-1+b1
      architecture: amd64
      missingdep: libboost-thread1.40.0 (>= 1.40.0-1)
     paths:
      -
       depchain:
        -
         package: libgnuradio-dev
         version: 3.2.2.dfsg-1
         architecture: all
         depends: libgnuradio (= 3.2.2.dfsg-1)
        -
         package: libgnuradio
         version: 3.2.2.dfsg-1
         architecture: all
         depends: libgnuradio-core0
        -
         package: libgnuradio-core0
         version: 3.2.2.dfsg-1+b1
         architecture: amd64
         depends: libgruel0 (= 3.2.2.dfsg-1+b1)
\end{verbatim}

The first part gives details about the package libgnuradio-dev, specifying its
status, source and architecture. The second part is the reason of the problem.
In this case it is a missing package that is essential to install
libgnuradio-dev. missindep is the dependency that cannot be satisfied is the
package libgruel0 , in this case: {\tt libboost-thread1.40.0 (>= 1.40.0-1)}.

The paths component gives all possible depchains from the root package
libgnuradio-dev to libgruel0 . Notice that we do not include the last node in
the dependency chain to avoid a useless repetition. Of course there might be
more then on path to reach libgruel0. Distcheck will unroll all of them.
Because of the structure of debian dependencies usually there are not so many
paths.

\begin{verbatim}
package: a
version: 1
depends: b, d

package: b
version: 1
depends: e

package: d
version: 1
depends: f

package: f
version: 1
conflicts: e

package: e
version: 1
conflicts: f
\end{verbatim}

The other possible cause of a problem is a conflict. Consider the following :

\begin{verbatim}
package: a
  version: 1
  status: broken
  reasons:
   -
    conflict:
     pkg1:
      package: e
      version: 1
     pkg2:
      package: f
      version: 1
     depchain1:
      -
       depchain:
        -
         package: a
         version: 1
         depends: b
        -
         package: b
         version: 1
         depends: e
     depchain2:
      -
       depchain:
        -
         package: a
         version: 1
         depends: d
        -
         package: d
         version: 1
         depends: f
\end{verbatim}

This is the general case of a deep conflict. I use an artificial example here
instead of a concrete one since this case is not very common and I was not able
to find one.

The first part of the distcheck report is as before with details about the
broken package. Since this is a conflict, and all conflicts are binary, we give
the two packages involved in the conflict first. Packages f and e are in
conflict, but they are not direct dependency of package a . For this reason, we
output the two paths that from a lead to f or e. All dependency chains for each
conflict are together. Again, since there might be more than one way from a to
reach the conflicting packages, we can have more then one depchain. 

\subsection{Understanding Explanations of installability}

%** We should fix the error codes (also for inconsistent input data)


%%%%%%%%%%%%%%%%%%%%%%%%%%%%%%%%%%%%%%%%%%%%%%%%%%%%%%%%%%%%%%%%%%%%%%%%%%
%  Copyright (C) 2010-2011  Pietro Abate <pietro.abate@pps.jussieu.fr>   %
%                           Ralf Treinen <ralf.treinen@pps.jussieu.fr>   %
%                           Unversité Paris-Diderot                      %
%                                                                        %
%  This documentation is free software: you can redistribute it and/or   %
%  modify it under the terms of the GNU General Public License as        %
%  published by the Free Software Foundation, either version 3 of the    %
%  License, or (at your option) any later version.                       %
%%%%%%%%%%%%%%%%%%%%%%%%%%%%%%%%%%%%%%%%%%%%%%%%%%%%%%%%%%%%%%%%%%%%%%%%%%

\section{Invocation}
\label{sec:invocation}

\debcheck{} accepts several different options, and also arguments.

\begin{alltt}
  \debcheck{} [option] ... [file] ...
\end{alltt}

All arguments are interpreted as filenames of Packages input files,
the contents of which go into the foreground. If no argument is given
then metadata of foreground packages is read from standard input.
In addition, one may specify listings of foreground packages with
the option \verb|--fg=<filename>|, and listings of background packages
with the option \verb|--bg=<filename>|. Input from files (but not from 
standard input) may be compressed with gzip or bzip2, provided \debcheck{}
was compiled with support for these compression libraries.

\begin{example}
We may check whether packages in \textit{non-free} are installable,
where dependencies may be satisfied from \textit{main} or \textit{contrib}:
\begin{verbatim}
dose-distcheck  -f -e \
    --bg=/var/lib/apt/lists/ftp.fr.debian.org_debian_dists_sid_main_binary-amd64_Packages\
    --bg=/var/lib/apt/lists/ftp.fr.debian.org_debian_dists_sid_contrib_binary-amd64_Packages\
    /var/lib/apt/lists/ftp.fr.debian.org_debian_dists_sid_non-free_binary-amd64_Packages
\end{verbatim}
\end{example} 

The initial distinction between foreground and background packages is
modified when using the \verb|--checkonly| option. This option takes
as value a comma-separated list of package names, possibly qualified
with a version constraint. The effect is that only packages that match
one of these package names are kept in the foreground, all others are
pushed into the background.

\begin{example}
\begin{alltt}
\debcheck{} --checkonly "libc6, 2ping (= 1.2.3-1)" Packages
\end{alltt}
\end{example}

There are also some filters on package metadata: \texttt{--latest}
keeps only the latest version of any package, and \texttt{--arch}
allows to filter out all packages that are not consistent with a
particular architecture (for instance, with \texttt{arch=amd64} only
packages with an Architecture value of \texttt{amd64} or \texttt{all}
are kept.

Other options controlling the output are explained in detail in
Section~\ref{sec:output}. A complete listing of all options can be found in
the \debcheck(1) manpage.



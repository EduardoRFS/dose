\subsection{Examples}

In the following examples, arch1 is declared as native architecture
and arch2 is the foreign architecture. Installed: yes means that the
package has to be considered as installed on the system.

\subsection{Multi Arch : None}

\begin{figure}
\begin{lstlisting}[style=debctrl,numbers=left,xleftmargin=20pt,basicstyle=\footnotesize\normalfont\ttfamily]
Package: a1
Architecture: arch1
Version: 1
Provides: a
Conflicts: a

Package: a1
Architecture: arch2
Version: 1
Provides: a
Conflicts: a
\end{lstlisting}
\caption{packages a1:arch1 and a1:arch2 cannot be installed together.}
\label{fig:arch-none}
\end{figure}

\subsection{Multi Arch : Allowed}

\begin{figure}
\begin{lstlisting}[style=debctrl,numbers=left,xleftmargin=20pt,basicstyle=\footnotesize\normalfont\ttfamily]
Package: a1
Architecture: arch1
Version: 1
APT-ID: 1
Multi-Arch: allowed
Installed: yes

Package: a1
Architecture: arch2
Version: 1
APT-ID: 1
Multi-Arch: allowed

Package: a2
Architecture: arch2
Version: 1
Depends: a1:any
Multi-Arch: same
\end{lstlisting}
\caption{Package a2:arch2 can be installed because the package a1:arch1
is currently installed.}
\label{fig:arch-allowed}
\end{figure}

\subsection{Multi Arch : Foreign}

\begin{figure}
\begin{lstlisting}[style=debctrl,numbers=left,xleftmargin=20pt,basicstyle=\footnotesize\normalfont\ttfamily]
Package: a1
Architecture: arch1
Version: 1
Multi-Arch: foreign

Package: a1
Architecture: arch2
Version: 1
Multi-Arch: foreign

Package: a2
Architecture: arch2
Version: 1
Depends: a1
\end{lstlisting}
\caption{Packages a1:arch1 and a2:arch2 are not coinstallable, but
package a2:arch2 can be installed}
\label{fig:arch-foreign}
\end{figure}



\subsection{Multi Arch : Same}

Packages a1:arch1 and a1:arch2 are coinstallable. Self conflicts are
ignored.

\begin{figure}
\begin{lstlisting}[style=debctrl,numbers=left,xleftmargin=20pt,basicstyle=\footnotesize\normalfont\ttfamily]
Package: a1
Architecture: arch1
Version: 1
Conflicts: a1
Multi-Arch: same

Package: a1
Architecture: arch2
Version: 1
Conflicts: a2
Multi-Arch: same
\end{lstlisting}
\caption{Packages a1:arch1 and a1:arch2 are coinstallable. Self
conflicts are ignored.}
\label{fig:arch-same-1}
\end{figure}

\begin{figure}
\begin{lstlisting}[style=debctrl,numbers=left,xleftmargin=20pt,basicstyle=\footnotesize\normalfont\ttfamily]
Package: a1
Architecture: arch1
Version: 1
Provides: a
Conflicts: a
Multi-Arch: same

Package: a1
Architecture: arch2
Version: 1
Provides: a
Conflicts: a
Multi-Arch: same
\end{lstlisting}
\caption{implicit self conflicts declared thought virtual packages are
ignored. Packages a1:arch1 and a1:arch2 are coinstallable}
\label{fig:arch-same-2}
\end{figure}

\begin{figure}
\begin{lstlisting}[style=debctrl,numbers=left,xleftmargin=20pt,basicstyle=\footnotesize\normalfont\ttfamily]
Package: a1
Architecture: arch1
Version: 1
Provides: a
Conflicts: a
Multi-Arch: same

Package: a1
Architecture: arch2
Version: 1
Provides: a
Conflicts: a
Multi-Arch: same

Package: a2
Architecture: arch2
Version: 1
Provides: a
Conflicts: a
Multi-Arch: same
\end{lstlisting}
\caption{packages a1:arch1 and a1:arch2 are not coinstallable with
package a2:arch2.}
\label{fig:arch-same-3}
\end{figure}

\begin{figure}
\begin{lstlisting}[style=debctrl,numbers=left,xleftmargin=20pt,basicstyle=\footnotesize\normalfont\ttfamily]
Package: a1
Architecture: arch1
Version: 1
Provides: a
Conflicts: a
Multi-Arch: same

Package: a1
Architecture: arch2
Version: 1
Provides: a
Conflicts: a
Multi-Arch: none
\end{lstlisting}
\caption{Coinstallability is possible only if all interested packages
are declared multi arch same.  Packages a1:arch1 and a1:arch2 are not
coinstallable}
\label{fig:arch-same-4}
\end{figure}



\section{Multi Archictecture in Debian}

In the Multi Architecture context, a Debian system has different notions of
architecture. The \texttt{Native} architecture is the architecture linked to
the package \texttt{dpkg} and is effectively chosen at install time by the user
(for example, a user can choose to have a native architecture i386 on an amd64
machine). \texttt{Foreign} architectures are a list of additional
architectures that are considered by dpkg and apt while resolving dependencies. 
In pre-Multi Architecture system, dependency resolution was restricted to
packages with the same architecture (with the exception of Architecture: all
packages), and a package of the same name but a different architecture was
assumed to not be co-installable. The Multi Architecture specification adds new
binary package field \texttt{Multi-Arch}, to be set on any package wanting to
be installed or used to satisfy dependencies on multiple architectures.

The Multi Architecture specify that if the field is present, the semantics are as follows:

\begin{itemize}
  \item{Multi-Arch: same} This package is co-installable with itself,
    but it must not be used to satisfy the dependency of any package
    of a different architecture from itself. 

  \item{Multi-Arch: foreign} The package is not co-installable with
    itself, but should be allowed to satisfy the dependencies of a
    package of a different arch from itself.

    If a package is declared Multi-Arch: foreign, preference should be
    given to a package for the native architecture if available; if it
    is not available, the package manager may automatically install
    any available package, regardless of architecture, or it may
    choose to make this an option controlled by user configuration. 

  \item{Multi-Arch: allowed} This permits the reverse-dependencies of
    the package to annotate their Depends: field to indicate that a
    foreign architecture version of the package satisfies the
    dependencies, but does not change the resolution of any existing
    dependencies. This value exists to prevent any packages from
    incorrectly annotating dependencies as being architecture-neutral
    without coordination with the maintainer of the depended-on
    package. 

  \item{Multi-Arch: none}
    The package can be installed and used to satisfy dependencies only
    on architectures equal to the arch from itself.
\end{itemize}

Moreover, a package for a foreign architecture is only installable if
all of its (recursive) dependencies are either marked as multiarch or
do not have corresponding packages installed for the native
architecture. An incomplete multiarch conversion for a given
dependency tree is equivalent to the status-quo. 

Since there are packages that can contain both architecture-specific
and architecture-agnostic components (python, contains that
interpreter that is architecture specific and modules that can be used
on any architecture), the multi-arch specification extend the
dependency relation language with a new attribute (\texttt{p:any})
that declares that a package's dependency on $p$ may be satisfied by a
package of any architecture, so long as the $p$ package declares
itself as \texttt{Multi-Arch: allowed}. 

A special exception are packages with \texttt{Architecture: all} that
are treated as equivalent to packages of the native architecture for
all dependency resolution. This means that for an Architecture: all
package to satisfy the dependencies of a foreign-architecture package,
it must be marked Multi-Arch: foreign or Multi-Arch: allowed. 

\subsection{Examples}

In the following examples, arch1 is declared as native architecture and arch2
is the foreign architecture. Installed: yes means that the package has
to be considered as installed on the system.

\subsection{Multi Arch : None}

\begin{figure}
\begin{lstlisting}[style=debctrl,numbers=left,xleftmargin=20pt,basicstyle=\footnotesize\normalfont\ttfamily]
Package: a1
Architecture: arch1
Version: 1
Provides: a
Conflicts: a

Package: a1
Architecture: arch2
Version: 1
Provides: a
Conflicts: a
\end{lstlisting}
\caption{packages a1:arch1 and a1:arch2 cannot be installed together.}
\label{fig:arch-none}
\end{figure}

\subsection{Multi Arch : Allowed}

\begin{figure}
\begin{lstlisting}[style=debctrl,numbers=left,xleftmargin=20pt,basicstyle=\footnotesize\normalfont\ttfamily]
Package: a1
Architecture: arch1
Version: 1
APT-ID: 1
Multi-Arch: allowed
Installed: yes

Package: a1
Architecture: arch2
Version: 1
APT-ID: 1
Multi-Arch: allowed

Package: a2
Architecture: arch2
Version: 1
Depends: a1:any
Multi-Arch: same
\end{lstlisting}
\caption{Package a2:arch2 can be installed because the package a1:arch1
is currently installed.}
\label{fig:arch-allowed}
\end{figure}

\subsection{Multi Arch : Foreign}

\begin{figure}
\begin{lstlisting}[style=debctrl,numbers=left,xleftmargin=20pt,basicstyle=\footnotesize\normalfont\ttfamily]
Package: a1
Architecture: arch1
Version: 1
Multi-Arch: foreign

Package: a1
Architecture: arch2
Version: 1
Multi-Arch: foreign

Package: a2
Architecture: arch2
Version: 1
Depends: a1
\end{lstlisting}
\caption{Packages a1:arch1 and a2:arch2 are not coinstallable, but
package a2:arch2 can be installed}
\label{fig:arch-foreign}
\end{figure}



\subsection{Multi Arch : Same}

Packages a1:arch1 and a1:arch2 are coinstallable. Self conflicts are
ignored.

\begin{figure}
\begin{lstlisting}[style=debctrl,numbers=left,xleftmargin=20pt,basicstyle=\footnotesize\normalfont\ttfamily]
Package: a1
Architecture: arch1
Version: 1
Conflicts: a1
Multi-Arch: same

Package: a1
Architecture: arch2
Version: 1
Conflicts: a2
Multi-Arch: same
\end{lstlisting}
\caption{Packages a1:arch1 and a1:arch2 are coinstallable. Self
conflicts are ignored.}
\label{fig:arch-same-1}
\end{figure}

\begin{figure}
\begin{lstlisting}[style=debctrl,numbers=left,xleftmargin=20pt,basicstyle=\footnotesize\normalfont\ttfamily]
Package: a1
Architecture: arch1
Version: 1
Provides: a
Conflicts: a
Multi-Arch: same

Package: a1
Architecture: arch2
Version: 1
Provides: a
Conflicts: a
Multi-Arch: same
\end{lstlisting}
\caption{implicit self conflicts declared thought virtual packages are
ignored. Packages a1:arch1 and a1:arch2 are coinstallable}
\label{fig:arch-same-2}
\end{figure}

\begin{figure}
\begin{lstlisting}[style=debctrl,numbers=left,xleftmargin=20pt,basicstyle=\footnotesize\normalfont\ttfamily]
Package: a1
Architecture: arch1
Version: 1
Provides: a
Conflicts: a
Multi-Arch: same

Package: a1
Architecture: arch2
Version: 1
Provides: a
Conflicts: a
Multi-Arch: same

Package: a2
Architecture: arch2
Version: 1
Provides: a
Conflicts: a
Multi-Arch: same
\end{lstlisting}
\caption{packages a1:arch1 and a1:arch2 are not coinstallable with
package a2:arch2.}
\label{fig:arch-same-3}
\end{figure}

\begin{figure}
\begin{lstlisting}[style=debctrl,numbers=left,xleftmargin=20pt,basicstyle=\footnotesize\normalfont\ttfamily]
Package: a1
Architecture: arch1
Version: 1
Provides: a
Conflicts: a
Multi-Arch: same

Package: a1
Architecture: arch2
Version: 1
Provides: a
Conflicts: a
Multi-Arch: none
\end{lstlisting}
\caption{Coinstallability is possible only if all interested packages
are declared multi arch same.  Packages a1:arch1 and a1:arch2 are not
coinstallable}
\label{fig:arch-same-4}
\end{figure}



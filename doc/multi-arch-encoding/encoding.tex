\section{Cudf 2.0 Encoding}

Debian packages can be represented by a tuple composed of name,
version, architecture, multiarch, depends, conflicts, provides.

The multi-arch encoding in relation with two factors : a reference
architecture, that we call ``native'' architecture and set of
``foreign'' architectures. For the same repository, we can have a
different cudf translation. This is in contrast with pre-multi
architecture cudf encoding of a Debian repository where one encoding
was used for all architectures. We define the set of foreign plus the
native architecture as $\cal{A}$.

Let \names{} be a set of names, \architectures{} a set of architecture
names and \versions{} be an ordered set of version strings. The exact
nature of versions are not important for our purpose safe the knowledge
that a closed set of Debian version can be safely associated with a
set cudf versions. We define the set of multi-architecture identifiers
as $\cal{M} = \{ \texttt{None, Same, Allowed,Foreign} \}$.

A \emph{constraint} is a unary predicate on versions: for any $v \in
\versions$ we may write constraints $=v$, $\neq v$, $<v$, $>v$, $\leq
v$, and $\geq v$.  The set of constraints is denoted by $\con$.

\begin{definition} 
  A \emph{package} $p$ is a tuple $(n,v,a,D,C)$ consisting of:
  \begin{itemize}
    \item a {\rm package name} $n \in \names$,
    \item a {\em version} $v \in \versions$,
    \item a {\em architecture name} $a \in \architectures{}$,
    \item a {\em multi-arch identifier} $m \in \multiarchs{M}$,
    \item a set of dependencies $D \in\power{\power{\names\times\con}}$,
    \item a set of conflicts $C \in\power{\names\times\con}$.
    \item a set of provides $C \in\power{\names\times \{ = \} }$.
  \end{itemize}
\end{definition}
where we use $\power{X}$ for the set of all subsets of $X$.

\begin{definition}
  For a set of packages $P$ we define
  \(\dep{P} := \bigcup_{d\in \dependency{p}} d\), and\linebreak
  \(\depnames{P} := \{ n | (n,c) \in\dep{P}\}\).
\end{definition}
That is, $\dep{P}$ is the set of all pairs of names and
constraints occurring in dependencies in $P$, and $\depnames{P}$ its
projection to names. We abbreviate $\dep{\{p\}}$ to $\dep{p}$.

A Debian repository is a set of Debian packages that are uniquely
identified by package name, version, and architecture name :
\begin{definition}
  \label{def:repository}
  A \emph{repository} is a set $R$ of packages such that for all $p, q
  \in R$: if $\name{p}=\name{q}$ and $\version{p}=\version{q}$ and 
  $\architecture{p}=\architecture{q}$ then $p=q$.
\end{definition}

We use the infix notation $(.)$ to access individual package
components (e.g.~$p.n$ stands for the name of package $p$). This
notion extends to sets of packages, e.g.~$R.n$ is the set of names of
packages in the repository $R$.

Provides encoding \ldots
\begin{definition}
  Given a repository of packages, we define the function 
  \[
    \rel{(n,c)} := \{ p \in R | p.n = n || \}
  \]
\end{definition}

In this section we will present the encoding of a Debian repository to
a cudf repository.

\paragraph{Uniqueness}.
Given a Debian repository, packages are uniquely identified by name,
version and architecture. Cudf packages, on the other hand are
identified by a tuple, namely $(name,version)$. To encode Debian
packages in cudf 2.0 without hampering readability, cudf packages are
written as $(name:architecture, version)$, therefore identifying the
corresponding cudf package by concatenating name with the architecture
using the character ``:'' as separator. \footnote{Note that since the
  character ``:'' is not part of the character sets used to denote
cudf packages, it will be encoded in a cudf document as its
hexadecimal encoding ``\%\ldots''.}

The multi architecture field in the debian stanza determines the
encoding of dependencies, conflicts and provides of the cudf package.

\paragraph{Dependencies : \texttt{None, Same, Allowed,Foreign}}
Given a Debian repository, and a package $p$ the set of dependencies
$p.D$\footnote{we are abusing notation here\ldots} is encoded in the
same way for all multi-architecture cases. For each constraint $(n,c)
\in p.D$ we create a cudf constraint of the form $(n:p.a,c)$. This
constraints enforce the fact that this dependency must be satisfied by
a package providing a name $n:p.a$.

To this general rule there are two exception. The first is that if the
dependency has the form $(n:any,c)$, then the we do not append the
package architecture : the dependency should be satisfied by ``any''
package providing the name $n$ (that is a package providing the name
$n:any$). Secondly, if the architecture of the package
is of type \texttt{all}, then we append to all dependencies $n$ then
native architecture. Formally we define the following function :

\begin{definition}
  Given a Debian package $p$, a native architecture, we define the
  function $\phi : \power{\names\times\con} ->
  \power{\names_{cudf}\times\con}$
\[
  \phi ((n,c)) = \left\{ 
    \begin{array}{rl}
      (n,c)           & \mbox{ if } n:any \\
      (n:na,c) & \mbox{ if } p.a = \tt{all} \\
      (n:p.a,c)       & \mbox{ otherwise }
    \end{array} 
  \right.
\]

The set of cudf dependencies by extending the function $\phi$ to the
set $p.D$.

\end{definition}

\paragraph{Conflicts : \texttt{None, Allowed,Foreign}}
Conflicts $p.C$, for all cases except for multi architecture {\texttt
Same}, are encoded appending the package architecture to all
$(n:p.a,c)$. Moreover, to enforce the flatness condition\footnote{to
be defined}, we also add an additional conflict with all the packages
that provides the name $p$.

\begin{definition}
  Given a Debian package $p$, a native architecture $na$ and a set of
  foreign architectures $\cal{F}$, we define the function $\gamma :
  \power{\names\times\con} -> \power{\names_{cudf}\times\con}$ :
\[
  \gamma ((n,c),a) = \left\{ 
    \begin{array}{rl}
      (n:na,c) & \mbox{ if } a = \tt{all} \\
      (n:a,c)        &  \mbox{ otherwise }
    \end{array} 
  \right.
\]

The set of cudf constraints is encoded as follows : 
\[
  C(p) = \{ p \} \cup \bigcup_{a \in \cal{F} \cup \{ na \} } \gamma(p.C,a)
\]
\end{definition}

\paragraph{Conflicts : \texttt{Some}}
Packages declared multi-arch \texttt{Some} are co-installable with
packages that have same version but different architectures, but they
must not be used to satisfy the dependency of any package of a
different architecture from itself. We encode these constraints by
first adding a self conflicts (that is a conflict with all packages of
any version and same architecture), a conflict with all packages with
same name but a different architecture and different version. Then for
each conflict $(n,c)$ and for each foreign architecture $a$ we add a
conflict to each real package $p$ providing the name $n$ of the form
$(p.name:a,c)$.

\begin{definition}
  Given a Debian package $p$, a native architecture $na$ and a set of
  foreign architectures $\cal{F}$, and let $\cal{A} = \{ na \}
  \cup \cal{F}$. We define the function $\gamma :
  \power{\names\times\con} -> \power{\names_{cudf}\times\con}$ :
\[
  \gamma ((n,c),a) = \left\{ 
    \begin{array}{rl}
      (n:na,c) & \mbox{ if } a = \tt{all} \\
      (n:a,c)        &  \mbox{ otherwise }
    \end{array} 
  \right.
\]

The set of cudf constraints is encoded as follows : 
\[
  \begin{array}{rl}
    C(p) = & \{ p \} \cup \\ 
    & \bigcup_{a \in \cal{A}} \{ (n:a:, \neq p.v) | (n,c) \in P.C \}\\
    & \bigcup_{a \in \cal{A}} \{ (\name{rn}:a:,c) | (n,c) \in \in
    P.C \mbox{ and } rn \in \rel{(n,c)} \} \\

  \end{array}
\]
\end{definition}


\paragraph{Provides : \texttt{None, Some}}

Provides are encoded appending the package architecture
to the package name as follows, with the exception of packages with
architecture \texttt{all}.

\[
  \phi ((n,c)) = \left\{ 
    \begin{array}{rl}
      n:native &  \mbox{ if } p.a = \tt{all} \\
      n:p.a & \mbox{ otherwise }
    \end{array} 
  \right.
\]
\[
  P(p) = \{ p \} \cup \gamma(p.P)
\]

\paragraph{Provides : \texttt{Foreign}}

A package that is declared \texttt{Multi-Arch: Foreign} cannot be
co-installed with a package with a same name, but different
architecture, but it can be used to satisfy the dependencies of
packages with different architectures. This constraint is reflected 
encoding provides as follows:

\[
  \begin{array}{rl}
    P(p) = & \{ p.name \} \cup \\
           & \{ \forall (n,c) \in p.P \phi((n,c), p.arch) \} \cup \\
           & \{ \forall a \in A \setminus \{ p.arch \}, \forall (n,c) \in p.P \phi(( n,c),a) \}
   \end{array}
\]


\paragraph{Provides : \texttt{Allowed}}
A package that is declared \texttt{Multi-Arch: Allowed} can be used to 
satisfy dependencies of type $:any$. The provides are encoded using
the following rule:

\[
  P(p) = \{ p.name \} \cup \{ p.name:any \} \cup \\
  \{ \forall (n,c) \in p.P \phi(( n,c), p.arch) \} \cup \\
  \{ \forall (n,c) \in p.P ( n:any,c) \} 
\]

The encoding takes care of enforcing co-installability w.r.t. the value of the
\texttt{MutliArch} field and to allow packages to correctly satisfy
dependencies of multi architecture enable packages.


\par{Co-installation}: For multiarch equal to \texttt{None}, \texttt{Foreign}
and \texttt{Allowed}, we prevent coinstallability with all other packages with
the same debian name by providing the the debian name (without the architecture
qualifier) and adding a self conflict on the debian name. For multiarch
\texttt{Same} we explicitly allow coinstallability and we do not add any
additional constraints.

\par{Satisfying other dependencies}. Form multiarch \texttt{None} and
\texttt{Same} the package can satisfy dependencies only of packages with the
same architecture. For multiarch \texttt{Foreign} we add one versioned provide
for each known foreign architecture. For \texttt{Allowed} we add a provide to
architecture \texttt{any}.


\section{Advanced Usage}

CUDF-based solvers understand user preferences and use them to select
a best solution. Compared with \aptget, this gives the user a greater
flexibility to define ``optimal'' solutions for installation problems,
for instance to minimize the number of new packages that are
installed, or to minimize the total installation size the packages to
upgrade.

\subsection{Defining your own optimization criteria}

Each CUDF solver implements a base optimization language, and some of
them implement extensions to this basic language to respond to
specific optimization requirements. \aptcudf, that is the bridge between
\aptget{} and the CUDF solver, associates to each \aptget{} command an
optimization criterion that can be either configured at each invocation
using one \aptget{} configuration option or by using the configuration
file (\texttt{/etc/apt-cudf.conf} ) of \aptcudf.

\begin{verbatim}
solver: *
upgrade: -new,-removed,-notuptodate
dist-upgrade: -notuptodate,-new
install: -removed,-changed
remove: -removed,-changed
trendy: -removed,-notuptodate,-unsat_recommends,-new
paranoid: -removed,-changed
\end{verbatim}

The field \texttt{solver} defines the (comma-separated) list of solvers
to which this stanza applies. The symbol ``*'' denotes that this
stanza applies to all solvers that do not have a specific stanza.

Each field of the stanza defines the default optimization criterion.
If one field name coincides with a standard apt-get action, like
install, remove, upgrade or dist-upgrade, then the corresponding
criterion will be used by the external solver. Otherwise, the field is
interpreted as a short-cut definition that can be used on the
\aptget{} command line.

Using the configuration option of \aptget{}
\texttt{APT::Solver::aspcud::Preferences}, the user can pass a
specific optimization criterion on the command line overwriting the
default. For example :

\begin{verbatim}
 apt-get -s --solver aspcud install totem -o "APT::Solver::aspcud::Preferences=trendy"
\end{verbatim}

\subsection{Pinning}

\subsubsection{Strict Pinning and Its Limitations}
When a package is available in more than one version, \aptget{} uses a
mechanism known as pinning to decide which version should be
installed. However, since this mechanism determines early in the
process which package versions must be considered and which package
versions should be ignored, it has also the consequence of
considerably limiting the search space. This might lead to \aptget{}
with its internal solver not finding a solution even if one might
exist when all packages are considered.

Anther consequence of the strict pinning policy of \aptget{} is that
if a package is specified on the command line with version or suite
annotations, overwriting the pinning strategy for this package, but
not for its dependencies, then the solver might not be able to
find a solution because not all packages are available. 

\subsubsection{Ignoring Pinning}
To circumvent this restriction and to allow the underlying solver to
explore the entire search space, \aptget{} can be configured to let the
CUDF solver ignore the pinning annotation.

The option \texttt{APT::Solver::Strict-Pinning}, when used in
conjunction with an external solver, tells \aptget{} to ignore pinning
information when solving dependencies. This may allow the external
solver to find a solution that is not found by the \aptget{}
internal solver.

\subsubsection{Relaxed Pinning}
Without relaxing the way that pinning information are encoded,
\aptcudf{} with an external CUDF solver would be effectively unable to
do better then \aptget{} because important information is lost on the
way. In order to overcome this limitation, \aptcudf{} has the ability
to reconstruct the user request and to use this information to provide
a possible solution. To this end, \aptcudf{} reads an environment
variable, named \texttt{APT\_GET\_CUDF\_CMDLINE} which the user can
pass along containing the invocation of \aptget.

To make it straightforward for the user, a very simple script called
\texttt{apt-cudf-get} is provided by the \aptcudf{} package.

\begin{verbatim}
#!/bin/sh
export APT_GET_CUDF_CMDLINE="apt-get $* -o APT::Solver::Strict-Pinning=\"false\""
apt-get $* -o APT::Solver::Strict-Pinning="false"
\end{verbatim}

The wrapper is invoked using the same commands as \aptget:

\begin{verbatim}
apt-cudf-get -s --solver aspcud install totem \
    -o "APT::Solver::aspcud::Preferences=-new,-changed"
\end{verbatim}
